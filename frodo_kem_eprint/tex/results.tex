\section{Results and Comparison}
In this section we discuss the results of our FPGA and microcontroller implementations and compare our implementations with others. In particular, we also compare with implementations of \textsf{NewHopeUSENIX}, even though comparing a standard lattice-based scheme with an ideal lattice-based scheme is not exactly an apples-to-apples comparison (as discussed in Section \ref{sec:ringvsstandard}). Our intention behind this comparison is to show the cost of removing the potential additional attack vector of ideal lattices (i.e., the ring structure).
\subsection{FPGA Results}

%\begin{table}[tbhp]
%\caption{Resource consumption of \textsf{FrodoKEM-cSHAKE} on a Xilinx Artix-7 XC7A35T FPGA.}
%\label{tab:res_fpga}
%\begin{center}
%\begin{tabular}{|c|c|c|c|c|c|c|}
%\hline
%%& \multicolumn{2}{c|}{\textsf{FrodoKEM-AES}} &\multicolumn{2}{c|}{\textsf{FrodoKEM-cSHAKE}} &&\\
%\textsf{FrodoKEM} Operation	& LUT/FF & Slice & DSP & BRAM & MHz & Ops/sec\\
%\hline
%\textsf{FrodoKEM-640} Keypair	& 6621/3511 & 1845 & 1 & 6 & 167 & 51 \\
%\textsf{FrodoKEM-640} Encaps	& 6745/3528 & 1855 & 1 & 11 & 167 & 51 \\
%\textsf{FrodoKEM-640} Decaps	& 7220/3549  & 1992  & 1 & 16 & 162 & 49 \\ \hline
%\textsf{FrodoKEM-976} Keypair	& 7155/3528 & 1981 & 1 & 8 & 167 & 22 \\
%\textsf{FrodoKEM-976} Encaps	& 7209/3537 & 1985 & 1 & 16 & 167 & 22 \\
%\textsf{FrodoKEM-976} Decaps	& 7773/3559 & 2158 & 1 & 24 & 162 & 21 \\ %\hline\hline
%%\textsf{NewHope-1024} Server \cite{oder2017implementing}	& 5142/4452 & - & 2 & 4 & 125 & 731 \\
%%\textsf{NewHope-1024} Client \cite{oder2017implementing}	& 4498/4635 & - & 2 & 4 & 117 & 653 \\ \hline
%%\textsf{LWE} Encryption \cite{HoweMORGB16_1} & 6078/4676 & 1811 & 1 & 73 & 125 & 1272 \\
%\hline
%\end{tabular}
%\end{center}
%\end{table}

Table \ref{tab:comp_fpga} provides post-place and route results of the proposed hardware designs, as well as comparative lattice-based cryptographic hardware designs. The 18Kb BRAM usage follows the requirements of the inputs of the operations, those being the public-key, the secret-key, and the ciphertext information. The increased of BRAM usage in Decaps results in a slight decrease in clock frequency. The area consumption of all the modules are similar, at least for each parameter set. This is essentially due to the reuse of the LWE multiplication core, which is reused for all vector-matrix multiplication and error addition operations. The increase between parameter sets is due to an increase from 640 to 976 in the matrix dimension, the rest of the design essentially remains the same.


\begin{table}[tbhp]
\caption{FPGA resource consumption of the proposed \textsf{FrodoKEM-cSHAKE} hardware design, with its sub-modules ($^*$), alongside \textsf{NewHopeUSENIX-1024} \cite{oder2017implementing} and \textsf{Standard-LWE} encryption \cite{DBLP:conf/dac/HoweMORGB16} hardware designs for comparison. \textsf{FrodoKEM} and \textsf{NewHopeUSENIX-1024} utilize a Xilinx Artix-7 FPGA and \textsf{Standard-LWE} utilizes a Xilinx Spartan-6 FPGA.}
\label{tab:comp_fpga}
\begin{center}
\resizebox{\textwidth}{!}{
\noindent\makebox[\textwidth]{
\begin{tabular}{|c|c|c|c|c|c|c|}
\hline
%& \multicolumn{2}{c|}{\textsf{FrodoKEM-AES}} &\multicolumn{2}{c|}{\textsf{FrodoKEM-cSHAKE}} &&\\
Cryptographic Operation	& LUT/FF & Slice & DSP & BRAM & MHz & Ops/sec\\
\hline
\textsf{FrodoKEM-640} Keypair 	& 6621/3511 & 1845 & 1 & 6 & 167 & 51 \\
\textsf{FrodoKEM-640} Encaps	& 6745/3528 & 1855 & 1 & 11 & 167 & 51 \\
\textsf{FrodoKEM-640} Decaps	& 7220/3549  & 1992  & 1 & 16 & 162 & 49 \\ \hline
\textsf{FrodoKEM-976} Keypair	& 7155/3528 & 1981 & 1 & 8 & 167 & 22 \\
\textsf{FrodoKEM-976} Encaps	& 7209/3537 & 1985 & 1 & 16 & 167 & 22 \\
\textsf{FrodoKEM-976} Decaps	& 7773/3559 & 2158 & 1 & 24 & 162 & 21 \\ \hline
cSHAKE$^*$	& 2744/1685 & 766 & 0 & 0 & 172 & 1.2m \\
Error+AES Sampler$^*$ & 1901/1140 & 756 & 0 & 0 & 184 & 184m \\

\hline\hline
\textsf{NewHopeUSENIX} Server \cite{oder2017implementing} 	& 5142/4452 & 1708 & 2 & 4 & 125 & 731 \\
\textsf{NewHopeUSENIX} Client \cite{oder2017implementing}	& 4498/4635 & 1483 & 2 & 4 & 117 & 653 \\ \hline
\textsf{LWE} Encryption \cite{DBLP:conf/dac/HoweMORGB16}  & 6078/4676 & 1811 & 1 & 73 & 125 & 1272 \\
\hline
\end{tabular}}}
\end{center}
\end{table}

Hardware results are also provided for the main components required; the error distribution sampler and the cSHAKE module. As per the specifications, the error sampler is combined with AES as a PRNG input to the lookup table. The large area consumption of this module is due to the use of AES, as well as employing a large number of comparators in order for high throughput. One cSHAKE module is used for generating the randomness for the matrix $\mathbf{A}$ and a second is used to generate the shared secret $\mathbf{ss}$ on-the-fly, which makes these cSHAKE modules the largest overall. The remaining area usage is consumed by control logic and the LWE multiplier; which requires a DSP for multiplication and a reasonable amount of LUTs for storage.


In Table \ref{tab:clks_fpga} we present the cycle counts for our FPGA designs. Clock cycle counts in Table \ref{tab:clks_fpga} are equivalent for the PRNG choice (either AES or cSHAKE) as this module runs in parallel to the vector-matrix multiplication within the LWE multiplication core, and does not affect the critical path of the operations (as described in Section \ref{sec:lwe_core}). The clock cycle counts for each cryptographic operation are defined by the MAC operations of all the matrix-matrix multiplications they require. Moreover, the multiplication of the largest matrices in each cryptographic operation, that is; $\mathbf{B} \leftarrow \mathbf{A} \mathbf{S} + \mathbf{E}$ for key generation, $\mathbf{B}^\prime \leftarrow \mathbf{S}^\prime \mathbf{A} + \mathbf{E}^\prime$ for encapsulation, or $\mathbf{B}^{\prime\prime} \leftarrow \mathbf{S}^\prime \mathbf{A} + \mathbf{E}^\prime$ for decapsulation, respectively contribute 100\%, 97.5\%, and 97.5\% to their overall clock cycle counts. As described in Section \ref{sec:frodo_fpga}, there is a one-time initialisation stage, for loading input information, initialising modules, and pre-storage matrices. This process lasts between 5.1k and 23.5k clock cycles, depending on the operation and parameter set used. This extra latency is not included in Table \ref{tab:clks_fpga} as it is negligible, even for one run of a \textsf{FrodoKEM} operation (at most, $< 0.5 \%$), and becomes even more so when averaged over numerous operations.

%For example in \textsf{FrodoKEM-640} encaps, this stage requires 5.1k clock cycles to load key information

\begin{table}[tbhp]
\caption{Clock cycle counts for our FPGA implementations of \textsf{FrodoKEM}.}
\label{tab:clks_fpga}
\begin{center}
\begin{tabular}{|l|r|r|r|r|}
\hline
& \multicolumn{2}{c|}{\textsf{FrodoKEM-AES}} &\multicolumn{2}{c|}{\textsf{FrodoKEM-cSHAKE}} \\
Operation	  & $n=640$ & $n=976$ & $n=640$ & $n=976$\\
\hline
%Operation 1					& xxx & xxx & xxx & xxx \\
%Operation 2					& xxx & xxx & xxx & xxx \\
%Operation 3					& xxx & xxx & xxx & xxx \\
Keypair							& 3,276,800 & 7,620,608 & 3,276,800 & 7,620,608 \\
Encaps							& 3,317,760 & 7,683,072 & 3,317,760 & 7,683,072 \\
Decaps							& 3,358,720 & 7,745,536 & 3,358,720 & 7,745,536 \\
\hline
\end{tabular}
\end{center}
\end{table}

Comparison results for related works is given in Table \ref{tab:comp_fpga}. The FPGA device used, the Xilinx Artix-7 XC7A35T FPGA, is similar to the one used by Oder and G{\"u}neysu \cite{oder2017implementing}, in order for a fair comparison. Although the \textsf{NewHopeUSENIX} design outperforms our proposed designs in terms of operations per second, the area consumption is comparable. The loss in throughput is expected and almost entirely due to the number of clock cycles required for each operation; \textsf{NewHopeUSENIX} requires 171k clock cycles for the server-side operations whereas \textsf{FrodoKEM} requires 3.3m. The increase in memory requirements is due to the differences in the key sizes, since \textsf{NewHopeUSENIX} uses polynomials instead of the matrices used in \textsf{FrodoKEM} (as mentioned in Section \ref{sec:ringvsstandard}).

The only other implementation of standard lattice-based cryptography in hardware is by Howe et al. \cite{DBLP:conf/dac/HoweMORGB16}, referred to as \textsf{Standard-LWE}, and is discussed here for comparison. The area consumptions are similar due to the similar LWE multiplication operations required, however, Howe et al. uses significantly smaller matrix dimensions in comparison to the \textsf{FrodoKEM} parameters, and hence we see an improvement. Moreover, their use of BRAM is significantly larger, due to precomputed keys, which we mitigate by using on-the-fly generation. Reusing keys is discussed by the authors of \textsf{FrodoKEM} \cite[Sec. 5.1.4]{frodo-nist} but is not recommended due to the potential attack vector it provides. Additionally, we also wanted to keep the memory requirements low and therefore decided to not store the entire matrix $\mathbf{A}$ in memory. The throughput performance of \textsf{Standard-LWE} is much higher in comparison to our research, due to a much lower clock cycle count of 98k required. This is essentially due to, again, the significantly smaller matrix dimensions and hence less multiplications required. Comparing the overall security targets of the schemes shows that the \textsf{Standard-LWE} implementation only provides 128 bits of \textit{classical} security, whereas our implementations provide 128 and 192 bits of post-quantum security.

%\textsf{Stanard-LWE} achieves such a low clock cycle count because the matrix $\mathbf{A}$ is precomputed and stored in memory, whereas in our implementation $\mathbf{A}$ is generated on-the-fly and clearly the bottleneck. 



\subsection{Microcontroller Results}
We use the pqm4 framework \cite{pqm4} to evaluate the proposed microcontroller implementation. In the framework, the running time of an operation is measured in cycle counts using libopencm3\footnote{\url{http://libopencm3.org/}}. The framework can also measure the stack usage with the help of stack canaries. Our development board runs at a clock frequency of 168 MHz.

In Table \ref{tab:res_micro} we show the cycle counts for the major building blocks of \textsf{FrodoKEM} as well as the entire key pair generation, encapsulation, and decapsulation. At 168 MHz, the key generation takes 266 ms, the encapsulation takes 284 ms, and the decapsulation takes 286 ms for \textsf{FrodoKEM-640-AES}. For \textsf{FrodoKEM-940-AES} the cycle counts are more than twice as high as for \textsf{FrodoKEM-640-AES}. The main reason for that is that the size of the matrix $\mathbf{A}$ grows quadratically when $n$ grows and the generation of $\mathbf{A}$ is the most time consuming part of our implementation. Further speed-ups could be achieved by speeding up the AES implementation. However, to the best of our knowledge, the implementation by Schwabe and Stoffelen \cite{DBLP:conf/sacrypt/SchwabeS16} that we used in our implementation is already the fastest published AES implementation. Some Cortex-M4-based microcontrollers also have access to an on-board hardware AES engine that would further speed up the AES. The implementations of \textsf{FrodoKEM-cSHAKE} are slower than \textsf{FrodoKEM-AES} implementations as cSHAKE is based on KECCAK and hardware platforms are where KECCAK really excels on, not software \cite{SHA3}. Therefore we expected \textsf{FrodoKEM-cSHAKE} to have a worse performance than \textsf{FrodoKEM-AES}.  

We furthermore see in Table \ref{tab:res_micro} that the $\mathbf{AS} + \mathbf{E}$ is the most time consuming part of the key pair generation as it accounts for 93\% of the run time for \textsf{FrodoKEM-640-AES}. Similarly $\mathbf{S}^\prime \mathbf{A} + \mathbf{E}^\prime$ is bottleneck for encapsulation and decapsulation (88\% resp. 87\%). The second most time-consuming operation is the sampling of a noise matrix. We measured the performance of noise sampling for matrices with dimension $n \times \bar{n}$. This operation is performed twice during every run of key pair generation, encapsulation, and decapsulation. It accounts for 6\% of the run time of each of the three algorithms (\textsf{FrodoKEM-640-AES}). In comparison to the computation of $\mathbf{AS} + \mathbf{E}$ or $\mathbf{S}^\prime \mathbf{A} + \mathbf{E}^\prime$, multiplications of smaller matrices cost much less cycles, as one can see in the cycle counts for $\mathbf{S}^\prime \mathbf{B} + \mathbf{E}^{\prime\prime}$.

\begin{table}[tbhp]
\caption{Cycle counts for our microcontroller implementation measured at a clock frequency of 168 Mhz.}
\label{tab:res_micro}
\begin{center}
\begin{tabular}{|l|r|r|r|r|}
\hline
& \multicolumn{2}{c|}{\textsf{FrodoKEM-AES}} &\multicolumn{2}{c|}{\textsf{FrodoKEM-cSHAKE}} \\
Operation	  & $n=640$ & $n=976$ & $n=640$ & $n=976$\\
\hline
$\mathbf{S}^\prime \mathbf{B} + \mathbf{E}^{\prime\prime}$	& 451,442  		& 687,728  		& 451,442  		& 687,728  \\
$\text{SampleMatrix}(\cdot,n,\bar{n},\cdot,\cdot)$					& 1,344,962  	& 1,480,483  	& 1,344,963  	& 1,480,484  \\
$\mathbf{AS} + \mathbf{E}$																	& 41,308,745 	& 96,035,515  & 82,256,529  & 181,809,613  \\
$\mathbf{S}^\prime \mathbf{A} + \mathbf{E}^\prime$					& 41,833,535  & 97,266,270  & 106,178,196 & 244,078,721  \\\hline
Keypair							& 44,603,160   & 101,273,066  & 85,585,315 & 187,070,653  \\
Encaps							& 47,742,966   & 106,933,956  & 112,103,350  & 253,735,550  \\
Decaps							& 48,051,929   & 107,393,295  & 112,442,770   & 254,194,895  \\
\hline
\end{tabular}
\end{center}
\end{table}

In Table \ref{tab:res_micro_mem} we list the peak stack usage for our implementations. For $n = 976$ the \textsf{FrodoKEM} specification \cite{frodo-nist} reports a peak stack memory usage of 189 kilobytes when using AES as PRNG and 156 kilobytes when using cSHAKE. The specification reports at most 81,836 bytes as static library size for non-vectorized implementations. As our development board has access to 192 kilobytes of RAM and one megabyte of Flash memory, the peak stack usage is what we focus on in the following. We managed to reduce these numbers so that the implementation comfortably fits onto the microcontroller and still leaves space for other applications running on it. For the AES-based implementation we reduced the memory consumption by 66\% and for the cSHAKE implementation we reduced it by 63\%.

\begin{table}[tbhp]
\caption{Stack usage in bytes for our microcontroller implementation.}
\label{tab:res_micro_mem}
\begin{center}
\begin{tabular}{|l|r|r|r|r|}
\hline
& \multicolumn{2}{c|}{\textsf{FrodoKEM-AES}} &\multicolumn{2}{c|}{\textsf{FrodoKEM-cSHAKE}} \\
Operation	  & $n=640$ & $n=976$ & $n=640$ & $n=976$\\
\hline
Keypair							& 23,396   & 35,484  & 22,376  & 33,800   \\
Encaps							& 41,292   & 63,484  & 37,792  & 57,968   \\
Decaps							& 51,684   & 63,628  & 48,184  & 58,112   \\
\hline
\end{tabular}
\end{center}
\end{table}


\begin{table}[tbhp]
\begin{center}
\caption{Comparison of our microcontroller implementation with other implementations.}
\label{tab:micro_comparison}
\begin{tabular}{|l|l|r|r|}
\hline
Implementation & Platform & Security Level & Cycle counts\\\hline
\textsf{FrodoKEM-976-AES} (\textbf{this work})	& Cortex-M4 & 192 bits  & 315,600,317\\
\textsf{FrodoKEM-640-cSHAKE} \cite{pqm4} & Cortex-M4 & 128 bits & 318,037,129 \\
\textsf{KyberNIST-768} \cite{pqm4} & Cortex-M4 & 192 bits & 4,224,704 \\
\textsf{NewHopeUSENIX-1024} \cite{alkim2016newhope}	& Cortex-M4 &255 bits & 2,561,438 \\
ECDH scalar multiplication \cite{DBLP:journals/dcc/DullHHHPSS15} &  Cortex-M0 & pre-quantum & 3,589,850\\
\hline
\end{tabular}
\end{center}
%\vspace{-0.7cm}
\end{table}

In Table \ref{tab:micro_comparison} we compare our implementation with other implementations of key exchange schemes on Cortex-M microprocessors. Our implementation of \textsf{FrodoKEM-976-AES} has a similar performance compared to the implementation of \textsf{FrodoKEM-640-cSHAKE} from the pqm4 library \cite{pqm4} even though the security level is higher (192 vs. 128 bits). Our implementation is two orders of magnitude slower than the \textsf{NewHopeUSENIX} implementation from \cite{alkim2016newhope}. The reason for that is that \textsf{NewHopeUSENIX} is based on ideal lattices that are inherently much more efficient as the main operation in ideal lattice-based cryptography is polynomial multiplication. Polynomial multiplication in integer rings can be efficiently realized with the number-theoretic transform that has a complexity of $O(n \log (n))$ while the matrix operations in \textsf{FrodoKEM} have a complexity of $O(n^2)$. Therefore a decently optimized implementation of a scheme based on ideal lattices will always be faster than any implementation of a scheme based on standard lattices targeting a similar security level. Furthermore, the implementation in \cite{alkim2016newhope} is only secure against chosen-plaintext attackers, not chosen-ciphertext attackers. Unsurprisingly the \textsf{KyberNIST-768} implementation from pqm4 also provides a better performance, as \textsf{KyberNIST-768} is a module lattice-based scheme that still retains some of the structure of ideal lattices and in particular also benefits from the speed-ups from the number-theoretic transform. The ECDH implementation of \cite{DBLP:journals/dcc/DullHHHPSS15} is also much more efficient. But as ECDH is not secure against attacks by quantum computers, it cannot be considered as alternative to \textsf{FrodoKEM}.

While being significantly slower than implementations of other schemes, \textsf{FrodoKEM} is also a very conservative choice that does not suffer from relying on a structured lattice as potential additional attack vector (like for instance \textsf{NewHopeUSENIX} does). Depending on the use case it might be sensible to use \textsf{NewHopeUSENIX} in scenarios that demand high efficiency and moderate security and use \textsf{FrodoKEM} in cases that have very high security requirements.